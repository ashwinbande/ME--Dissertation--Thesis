A group signature scheme is an improved adaptation of digital signature scheme, which uses group based authentication. In group based authentication system, the privacy of the signer is maintained while providing authentication and integrity of the signed message. In those schemes, a signer can sign a message on behalf of the group to which he belongs. The verifier of the signature can only verify the signed message as, is it signed by an authentic group member or not, but cannot identify the individual group member, who was the signer of the message. Thus protecting the group members anonymity while providing authentication to the verifier. In some cases of dispute, identification of the signer can be revealed only by authenticate opening entity, possessing a special opening key.
 
A group signature scheme needs to be efficient, robust and secure with resistant to various attacks. The scheme must have properties like correctness, unforgeability, anonymity, traceability, etc. The proposed scheme provides a different approach for privacy management of signer and also provides full revocation support for any member at any time. The scheme is also fully dynamic and can add any member at any time after creation of the group. 
 
This scheme also reduces a significant computational cost of producing the signature, compared to other schemes. This scheme uses verifier-local revocation technique, which is found to be secure and most efficient among other methods and also provide a suitable solution for the significant computational overhead required for verifier-local revocation process. This group signature scheme is proved to be secure and coalition resistant under the Diffie Hellman assumption and strong RSA assumption, which are widely accepted cryptographic assumptions.