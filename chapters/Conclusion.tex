In this dissertation, we have proposed an efficient method to implement a dynamic group signature scheme with distributed authorities. The scheme also provides the verifiable opening of the signature. The implemented group signature scheme also enables the secure revocation of group members by using the local verifier revocation technique. The problem of linear incrementation of time required for verification in local verifier revocation is also tackled with an efficient solution of resetting the group. This scheme enhances the privacy of the group members and provides an effective way of identity management. The scheme is practical to implement and proved to be secure under the Diffie Hellman assumption and strong RSA assumption. The scheme also found to be resistant against various attacks like framing, forging or coalition of members. The presented scheme also required less computational overhead compared to other similar schemes. We have shown the effectiveness of our scheme by comparing with state of the art ACJT group signature scheme and almost all important group signature scheme.

\subsubsection{Future Scope}\index{Future Scope}
In future, it can be possible to extend the scheme to practical applications like e-voting and e-bidding. It is also plausible to implement this scheme as a replacement of digital signatures in conceal organizational structures like a company or government departments to provide secure authentication of documents while preserving the privacy of the signer. It is feasible to optimize the revocation technique of the scheme by using some other cryptographic concepts so that the scheme may provide full anonymity instead partial anonymity.