%Problem Definition
This dissertation deals with the problem of developing an architecture for a group signatures scheme which implements the functionalities such that:
\begin{itemize}
\item The distribution of the authorities separates the responsibility of the Group Manager into different entities in such a way that each authority only have limited power over the trusted tasks.
\item The trusted authorities should be distributed in such a way that it enables them to execute their duties without any requirement of any response from the other authorities but not able to perform any task allocated to the other authorities.
\item The architecture should implement a secure and trusted procedure for dynamic addition as well as revocation of the group members.
\item The revocation of the members should be carried out in such a way that the past messages, as well as the messages signed by the revoked member in future, should be identified as invalid.
\item The computational time and cost for the generation of the signature should be lessened compared to similar group signature schemes.
\item The implementation must be resistance to multiple attacks like coalition attack and profiling attack etc.
\item The scheme must satisfy the definition of Bellare's strict model \cite{bellare2003foundations} as well as provide security properties like correctness, anonymity, unlinkability, and traceability, etc. from the Bellare's model. 
\end{itemize}